\section{Überschrift 3}

    Satz3, Satz3, Satz3, Satz3 Satz3, Satz3, Satz3, Satz3 Satz3, Satz3, Satz3, Satz3 Satz3, Satz3, Satz3, Satz3 Satz3, Satz3, Satz3, Satz3 
    Satz3, Satz3, Satz3, Satz3 Satz3, Satz3, Satz3, Satz3 Satz3, Satz3, Satz3, Satz3 Satz3, Satz3, Satz3, Satz3 Satz3, Satz3, Satz3, Satz3 
    Satz3, Satz3, Satz3, Satz3 Satz3, Satz3, Satz3, Satz3 Satz3, Satz3, Satz3, Satz3 Satz3, Satz3, Satz3, Satz3 .


    Satz3, Satz3, Satz3, Satz3 Satz3, Satz3, Satz3, Satz3 Satz3, Satz3, Satz3, Satz3 Satz3, \\
    Satz3, Satz3, Satz3 Satz3, Satz3, Satz3, Satz3 
    Satz3, Satz3, Satz3, Satz3 Satz3, Satz3, Satz3, Satz3 Satz3, Satz3, Satz3, Satz3 Satz3, Satz3, Satz3,
    
    \paragraph*{Absatz 1}
    Satz3 Satz3, Satz3, Satz3, Satz3.

    \subsection{Überschrift 3.1}
    Satz3, Satz3, Satz3, Satz3 Satz3, Satz3, Satz3, Satz3 Satz3, Satz3, Satz3, Satz3 Satz3, \\
    Satz3, Satz3, Satz3 Satz3, Satz3, Satz3, Satz3 
    Satz3, Satz3, Satz3, Satz3 Satz3, Satz3, Satz3, Satz3 Satz3, Satz3, Satz3, Satz3 Satz3, Satz3, Satz3,

    \begin{description}
        \item [Paçoca] Mein Hund. Mein Hund.Mein Hund.Mein Hund.Mein Hund.Mein Hund.Mein Hund.Mein Hund.Mein Hund.Mein Hund.
        \item [Ireni] Meine Mutter. Meine Mutter.Meine Mutter.Meine Mutter.Meine Mutter.Meine Mutter.Meine Mutter.Meine Mutter.
    \end{description}


    Ein Bild von Pacoca.
    \begin{figure}[htb]
        \centering
        \includegraphics[width=0.50\textwidth]{/home/bruno/git/LaTex/sonopacoca.jpeg}
        \caption{Ein Bild von Pacoca}
        \label{fig:pacoca}
    \end{figure}

    \section{Überschrift 4}

    Hier wollen wir Tabelle hinzufüegen.
    \begin{table}
        \centering
        \begin{tabular}{|c|c|c|c|}
            \textbf{Farbe} & \textbf{Form} &   \textbf{Zahl} & \textbf{Hund} \\
            Rot             & Rechteck      &  100           & Pacoca  \\
            Blau            & Kreis         &  23            & Nina  \\
            Gelb            & Dreieck       &  44            & Bino  \\
            Lila            & Rechteck      &  66            & Boby  \\
            Gelb            & Kreis         &  88            & Amora  \\
        \end{tabular}
        \caption{Eine dumme Beschreibung der Tabelle}
        \label{tbl:beispieltabelle1}
    \end{table}


    \begin{table}
        \centering
        \begin{tabular}{|lccr|}
            \hline
            \textbf{Farbe} & \textbf{Form} &   \textbf{Zahl} & \textbf{Hund} \\
            Rot             & Rechteck      &  100           & Pacoca  \\
            \hline
            Blau            & Kreis         &  23            & Nina  \\
            \hline
            Gelb            & Dreieck       &  44            & Bino  \\
            \hline
            Lila            & Rechteck      &  66            & Boby  \\
            \hline
            Gelb            & Kreis         &  88            & Amora  \\
            \hline
        \end{tabular}
        \caption{Diese zweite ist auch nicht besser.}
        \label{tbl:beispieltabelle1}
    \end{table}